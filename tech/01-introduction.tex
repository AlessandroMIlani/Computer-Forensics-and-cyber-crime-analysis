\chapter{Introduction}

\section{Topics}
\begin{itemize}
    \item \textbf{Forensics Analysis} \\
        use of logic and meaningful knowledge and methodological approach to 
        legal problems and criminal investigation.
    \item \textbf{Computer Forensics} \\
        Collection, preservation and analysis of digital evidence 
        (inside file system, email, cloud account etc...) to support 
        investigation and legal proceedings
\end{itemize}

\section{Forensics History}

\subsection{Ancient Times}
Forensic science dates back to \textbf{Babylon (1900 BC)} where fingerprints 
were used for identification, and \textbf{China (1248 AC)} with forensic pathology.
 In the \textbf{UK (1835)}, bullet comparison solved a case, and by \textbf{1892}, 
 the first murder was solved using fingerprints.

\subsection{Modern Times}
Forensic standards grew in police departments, with the first crime lab in 
\textbf{1923}. DNA fingerprinting began in the \textbf{1950s}, and DNA profiling
 was developed by \textbf{1985}. \textbf{AFIS} systems emerged in the late 
 \textbf{1980s}. Today, AI, toxicology, and digital forensics are key areas 
 of innovation.

\subsection{Digital Field}

\subsubsection{Early Times}
In \textbf{1989}, Robert Morris was convicted under the Computer Fraud and 
Abuse Act, marking the first use of computer logs in forensics. That year, 
\textbf{IACIS} was founded, followed by \textbf{IOCE} in \textbf{1995} to 
share digital forensic practices.

\subsubsection{Recent Times}
\begin{itemize}
    \item \textbf{1990}: Forensic tools like EnCase emerged
    \item \textbf{2000}: Digital forensics became widespread in law enforcement
    \item \textbf{2010}: Growth of cloud and mobile forensics, automation, and 
    machine learning
    \item \textbf{2020}: Advances in crypto, blockchain, and AI improve digital 
    forensics
\end{itemize}

\section{Computer Forensics Definitions}

\textbf{US\_CERT:} The discipline that combines elements of law and computer 
science to collect and analyze data from computer systems, 
networks, wireless communications, and storage devices in a 
way that is admissible as evidence in a court of law. \\
\textbf{A. Ghirardini -Computer Forensics:} The discipline whose goal is preservation, identification, analysis of 
information system to the aim of identification of evidences during 
investigation activities. \\
\textbf{NIST glossary:} The application of computer science and investigative procedures 
involving the examination of digital evidence - following proper search 
authority, chain of custody, validation with mathematics, use of validated 
tools, repeatability, reporting, and possibly expert testimony.

\section{CF purpose(s)}

\subsection{CF Q\&A}
During an investigation, digital forensics need to analysis data to answer some key questions: \\
\begin{minipage}[t]{0.45\textwidth}
\begin{itemize}
    \item \textbf{What happened?}
    \item \textbf{Who was involved?}
    \item \textbf{When did it take place?}
\end{itemize}
\end{minipage}
\begin{minipage}[t]{0.45\textwidth}
\begin{itemize}
    \item \textbf{Where did it take place?}
    \item \textbf{Why did it take place?}
    \item \textbf{How did an incident occur?} \newline
\end{itemize}
\end{minipage} 

The answers to these questions are essential for support legal proceedings and mitigate 
possibility of future incidents with a preventive approach.

\subsection{CF Goals}

The goals of computer forensics (CF) are multifaceted and aim to provide a
comprehensive understanding of digital incidents. \\
Firstly, CF seeks to retrieve what has been the input, such as what has been typed. 
It also aims to determine the actions performed, for example, what programs have been run 
and what peripherals have been connected. \\
Additionally, CF involves analyzing used files to understand what modifications 
have been done and when these modifications occurred (and the information from an OS 
are not enougth, because are an abstraction managed by the file system $\rightarrow$ needed 
bit analysis (like for ereased data)). \\ 
Another critical goal is to identify the damage done, 
such as what data have been erased. In essence, the overarching goal 
of CF is to \textbf{gain a technical comprehension of what happened during the 
incident} (from a technical point of view).
