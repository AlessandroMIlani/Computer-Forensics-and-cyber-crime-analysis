\section{File System Forensics Domain}
File system forensics is a domain focused on analyzing the structure and organization of file systems to uncover hidden or deleted data. It involves several key methodologies and tools for efficient investigation.

\subsection{Key Points in File System Forensics}
\begin{itemize}
    \item \textbf{Slack Space Analysis:} Investigates unused portions of disk sectors, which may contain residual data from previous operations.
    \item \textbf{File Carving:} Involves the extraction of deleted files by analyzing residual data clusters.
    \item \textbf{Registry and OS Configuration Analysis:} Recovers user and system activities by examining data such as the Windows Registry or configuration files (e.g., \texttt{/etc}, \texttt{\~/.config}, \texttt{\~/.bashrc}).
\end{itemize}

\subsection{Relevant Tools}
\begin{itemize}
    \item \textbf{Low-Level Tools:} \texttt{stat}, \texttt{istat}, \texttt{debugfs}.
    \item \textbf{High-Level Tools:} FTK Imager, Autopsy.
\end{itemize}

\section{File System Overview}
\subsection{Definition and Functionality}
A file is the smallest logical unit of storage from the user’s perspective and can store data in formats such as bytes, lines, or records. Logical files are mapped into physical entities like computer RAM, hard drives, or cloud storage by the operating system.
\begin{itemize}
    \item File systems define the organization of files on computing devices.
    \item Operating systems provide rules to read, write, and manage data.
\end{itemize}

\section{File Attributes}
Files have various attributes that define their characteristics and usage:
\begin{itemize}
    \item \textbf{Name:} A human-readable identifier.
    \item \textbf{Type:} Categorizes how the file is manipulated (e.g., based on a "magic number" at the file start). Note that Windows often relies on file extensions for application association.
    \item \textbf{Protection:} Specifies access control, which varies by OS and file system. For example:
        \begin{itemize}
            \item Unix-like systems use owner/group permissions (read, write, execute).
        \end{itemize}
    \item \textbf{Location:} Indicates physical or logical storage.
    \item \textbf{Size:} Represents the file's data size.
\end{itemize}

\section{File System Formatting}
Formatting prepares storage devices for data storage by configuring them with specific file system structures. It can involve:
\begin{itemize}
    \item Erasing existing data.
    \item Creating partitions and formatting each partition separately (e.g., NTFS for Windows, APFS for macOS, ext4 for Linux).
    \item Establishing foundational structures like root directories, FAT/inode tables, and superblocks.
\end{itemize}

\subsection{Types of Formatting}
\begin{itemize}
    \item \textbf{Full Formatting:} Slow process erasing all sectors and marking bad ones.
    \item \textbf{Quick Formatting:} Clears file system tables without erasing data clusters.
\end{itemize}

\section{FAT File System}
\subsection{Initialization}
When a hard drive is formatted with FAT:
\begin{itemize}
    \item A boot record is created, storing OS details and disk characteristics.
    \item Two copies of the Master File Table are generated for redundancy.
    \item A directory table is created, containing information about top-level folders and files.
\end{itemize}

\subsection{Structure}
The FAT (File Allocation Table) records file positions and resides near the start of the volume. Two copies are typically maintained for redundancy.

\subsection{Data Representation in FAT}
\begin{itemize}
    \item Directory entries contain attributes like file names, start cluster numbers, and file sizes.
    \item The table links clusters using a linked list, marking the end of a file with \texttt{EOF}.
\end{itemize}

\section{FAT Pros and Cons}
\subsection{Advantages}
\begin{itemize}
    \item \textbf{Portability:} FAT is compatible with multiple operating systems.
    \item \textbf{Migration:} Easy to transition to more advanced file systems like NTFS.
    \item \textbf{Performance:} Efficient for small volumes (a few GB) due to its lightweight metadata structure.
\end{itemize}

\subsection{Disadvantages}
\begin{itemize}
    \item Lack of advanced features like on-the-fly compression or user quotas.
    \item Inefficiency with large numbers of files due to fragmentation and a linked-list structure.
    \item No built-in security features, such as encryption or access control lists.
\end{itemize}

\section{References for FAT Forensics}
\begin{itemize}
    \item Microsoft Extensible Firmware Initiative FAT32 File System Specification.
    \item FAT recovery methods, such as identifying entries beginning with \texttt{0xe5}.
    \item Resources:
        \begin{itemize}
            \item \url{https://download.microsoft.com/.../fatgen103.doc}
            \item \url{https://forensics.wiki/fat/}
        \end{itemize}
\end{itemize}

\section{FAT References}
For further details about the FAT file system, the following references provide comprehensive technical specifications and forensic guidance:
\begin{itemize}
    \item \url{https://download.microsoft.com/.../fatgen103.doc}
    \item Microsoft Extensible Firmware Initiative FAT32 File System Specification.
    \item \url{https://forensics.wiki/fat/}
    \item FAT recovery techniques include identifying entries starting with \texttt{0xe5}.
\end{itemize}

\section{File Copy and Cloning}
\subsection{Challenges in File Copying}
While normal file copy commands preserve file content, they alter metadata attributes like creation dates. For accurate forensic duplication, a bit-by-bit copy is necessary.
\subsection{Techniques}
\begin{itemize}
    \item \texttt{dd} creates exact data dumps, including metadata.
    \item Variants like \texttt{dcfldd} and \texttt{dc3dd} add features like on-the-fly hashing and progress reporting.
\end{itemize}
\subsection{Examples}
\begin{itemize}
    \item Clone a hard drive: \texttt{dd if=/dev/sda of=/dev/sdb}
    \item Save a hard drive to an image file: \texttt{dd if=/dev/hda of=/image.img}
    \item Compress an image in 100 MB blocks: \texttt{dd if=/dev/hda bs=100M | gzip -c > /image.img}
    \item Overwrite a hard drive: \texttt{dcfldd pattern=00 vf=/dev/hdb}
\end{itemize}

\section{File Identification}
\subsection{Techniques}
\begin{itemize}
    \item File extensions are unreliable as they can be modified.
    \item Metadata or the first bytes of a file (signature) often provide reliable identification.
\end{itemize}
\subsection{Resources}
\url{https://en.wikipedia.org/wiki/List_of_file_signatures}

\section{Metadata in File Systems}
Metadata includes critical details such as file name, ownership, permissions, size, timestamps, and recovery data. These details vary across file systems like FAT32, NTFS, ext2/ext3/ext4.

\section{Slack Space}
\subsection{Definition}
Slack space refers to unused space within a sector allocated to a file, caused by mismatches between file size and sector size (e.g., a 392-byte file in a 512-byte sector leaves 120 bytes as slack).
\subsection{Significance}
Residual data in slack space can remain even after files are deleted, creating a potential source of evidence.

\section{Recovery Processes}
\begin{itemize}
    \item Analyze file system structures like the Master File Table (MFT) for metadata.
    \item Metadata timelines can be essential in legal cases to verify evidence integrity.
    \item Techniques include recovering orphaned or deleted files using file signatures.
\end{itemize}

\section{File Analysis Through Metadata}
Metadata reveals details about a file’s creation, modification, and context. It can expose hidden information, correlate related data, and aid forensic investigations.

\section{Metadata Examples}
\subsection{ODF}
Open Document Format files store metadata in a separate XML file within a compressed archive.
\subsection{JPEG}
EXIF metadata in JPEG files contains details such as camera make/model, resolution, software version, and timestamps.

\section{Hexadecimal Tools}
\begin{itemize}
    \item \texttt{hexdump:} Displays a file in hexadecimal format.
    \item \texttt{hexedit:} Command-line tool for hexadecimal viewing and editing.
    \item \texttt{ghex:} GNOME-based tool for graphical editing in hex or ASCII.
\end{itemize}

\section{Data Recovery Tools}
\subsection{Foremost}
A command-line tool for carving deleted files from disk images by identifying file types via signatures.
\subsection{Photorec}
Scans storage media directly to recover deleted files, bypassing file systems. It is effective for damaged or formatted drives.

\section{File Deletion and Recovery}
Deleted files often retain fragments in slack space or unallocated clusters. Forensic techniques can recover these fragments even after deletion.

\section{Data Sanitization}
\subsection{Tools}
\begin{itemize}
    \item File shredder programs overwrite selected files to prevent recovery.
    \item Data destruction software erases entire storage devices using multiple overwrites.
\end{itemize}
\subsection{Procedures}
The USAF’s AFSSI-5020 requires three overwrite passes:
\begin{itemize}
    \item First pass: Write zeros.
    \item Second pass: Write ones.
    \item Third pass: Write random characters and verify.
\end{itemize}
