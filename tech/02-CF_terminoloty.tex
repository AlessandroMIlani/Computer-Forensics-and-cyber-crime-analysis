\chapter{CF Terminology \& revelant conceps}

\section{Terms}

\subsection{Digital evidence}
Technical definition of Digital evidence is very similar to the legal ones:
\begin{boxH}
    Data stored or transmitted in digital form that can be used in court.
\end{boxH}
 
The conrnerstones of digital forensics are the different levels of \textbf{abstraction}, requires 
\textbf{interpretation}, are \textbf{fragile}, may be \textbf{voluminous} 
and the difficulty to discover \textbf{connection} between
data and reality (the connection need to be done beafore entering the court). \\

Digital evidence also requires a deep technical understanding of the possible types of data 
(files, emails, logs, metadata) and the legal requiremnets for each of one to collect and preserve it. 
(To make all of this effective, knowledge of file systems, network protocols 
and encryption are essential) \\


\subsection{Chain of custody}
\begin{boxH}
    Documented and \textbf{unbroken} process of handling evidence 
    from the time it is collected until it is presented in court
\end{boxH}

This procedure is essential to ensure the integrity of the evidence and to avoid 
that them to be tampered or accessed by unauthorized people. \\
Keep the chain of custody requires knowledge about how to document evidence 
collection, storage, and access (logging procedures, secure storage, legal prtocols etc..) \\

If the chain of custody is \textbf{broken}, the evidence may be considered \textbf{inadmissible in court}
(so is needed know the regulation of the state to decide how to manage it). 

\subsection{Data acuisition}
\begin{boxH}
    The process of collecting digital evidence from devices  
    without altering or damaging the original data.
\end{boxH}

One of the biggest problem fo the managment of digital evidence, because it's needed to be performed
on hostile systems (that can be infected, compromised, have a malware or system to avoid copy
like edited system call for make other program to fail). in a not coltrolled envorment like a crime scene. \\
So are needed knowledge of disk imaging and live data capture in order to not alter what's going on 
on the suspect system. Are also required expertise in forensics acquisition, analysis 
tools (like FTK Imager, EnCase) and knowledge of file systems, write-blockers, and hashing 
(crucial for ensuring integrity).


\subsection{Hashing}
\begin{boxH}
    The process of converting data into a fixed-length string of 
    bits, which represents the data uniquely
\end{boxH}

It's used in the chain of custody for ensure the integrity of the digital evidence and so verity
that a file has not been altered. \\
Require understanding of hashing algorithms (strengths and 
weaknesses, e.g. MD5 collision), formats 
(hex, base64 etc...) (if wrong formats are sued, the chain of custody is broken and each information
gathered from that point is considered not valid) 
and expertise in hashing tools (sha256sum, hashdeep, FTK imager, Autopsy). \\

Have to be used any time an evidence is "managed" (copied, moved)


\subsection{Write Blocker}
\begin{boxH}
    Hardware or software tool used to prevent any data from 
being written to a storage device during analysis, preserving 
the original data content
\end{boxH}

To be operated, require understanding of how write-blocking devices work 
and how they can be implemented in forensic procedures. \\ 

It's essential for the legally defensible acquisition.


\subsection{Forensic image}

\begin{boxH}
    A bit-by-bit copy of digital media, including deleted files and 
    data in slack space, which is an exact replica of the original device
\end{boxH}

The goal of a forensic image is to preserve the original evidence and aovid the modification
of the original data. \\

To be performed in a correct way, requires understanding of mechanisms to copy information 
in digital devices (file system knowledge and behavior) and familiarity with  with bit-by-bit copy tools
(DD, FTK Imager, EnCase, Guymager). \\

As the hashing, it's need to be used any time an evicene is "managed" (copied, moved)

\section{Scenarios}
There are some possible scenario that a computer forensic investigator can face:
\begin{itemize}[itemsep=0pt]
    \item Internet abuse from employee
    \item computer-aided frauds
    \item Data unauthorized manupulation (theft or destruction)
    \item Computer/network manage assessment
    \item \dots any other case that include digital evidence
\end{itemize}

\section{investigation phases}

A Computer forensics investigatos usually follow standard phases that guide him. There are different standards like: NIST family, ACPO guidelines (UK), ISO/IEC 27042, SWDGE.

\subsection{Phases}

\begin{itemize}[itemsep=0pt]
    \item \textbf{identification:} When the investigator come for the first time to the crime scene and need to identify potential source of relevant digital evicences. 
    
    \item \textbf{collection:} The letteral pick up of the evidence (like a computer or a smartphone) or a remote taking possession of the evidence (like for a remote server) and its connection (e.g. network or physical, like USB disk). \\ It's splitted from acquisition because it's a critical phase where the evidence can be altered and lost utility for the investigation (es. data corruption, lost of metadata etc...)
    
    \item \textbf{acquisition:} Electronically retrieving data by running various CF tools and software suites
    
    \item \textbf{evaluation:} Evaluating the data recovered to determine if and how it could be used against the suspect (e.g. for prosecution in court)
    
    \item \textbf{presentation:} Presenting the evidence discovered in a manner which is suitable for lawyers, non-technical staff/management and the law (and internal rules)
\end{itemize}

\subsection{Identification}
During the identification phase is important \textbf{recognize} all the \textbf{relevant data sources} before any acquisition, even if no physical present, like data in the cloud \\ A imple \textbf{list of example} are: hard drives (HDD/SSD), memory (RAM), mobile devices (smartphones, tablets), cloud storage, network traffic, removable media (USB drives, DVDs), IoT devices and embedded systems (like smart washing machines) \\ 
\bigskip For identify these sourcer, the investigator can perform some actions, like:
\begin{itemize}[itemsep=0pt]

    \item Perform an initial survey of the scene (physical or network environment)
    
    \item Identify key devices and data locations (local storage, remote servers, cloud services)
    
    \item Check for connected devices, including peripherals like printers, removable media, or network-attached devices
    
    \item Map all potential data sources using network topology diagrams or asset inventories
    
\end{itemize}

A particolar aspect that need to be considered is the possible present of "ephemeral" storage or data, like cloud syncing, hidden secttor, tmp, dat in ram etc...

\subsection{Collection}
During the collection, the focus is on gathering evidence from identified data sources while ensuring the preservation of its integrity. An important key point is the implementation of methods that \textbf{minimize the risk of evidence tampering or data loss}. \\
For enforce this key point, it's importnat \textbf{isolate devices} to prevent them from being tampered with remotely (e.g., disconnect them from the network), use devices to \textbf{block external
communication} for mobile or wireless devices and use network isolation tools for virtual and cloud environments to prevent remote access (like use a virtual private cloud). \\
A particular note is for the managment of live systems where is needed ensure evidence integrity while
maintaining system uptime (so not shoutting down the system for avoid the loss of volatile data). \\
Create a \textbf{detailed record} of the condition and state of the evidence
\begin{itemize}[itemsep=0pt]
    \item take photographs of the devices in situ, including connected peripherals and the physical state
    \item  record serial numbers, device models, and any other identifiable information
    \item  document the scene, noting which devices were running, whether screens were active o locked, and any other visible indicators
\end{itemize}
\textbf{hint:} complete documentation is crucial to prevent legal challenges regarding the integrity of the evidence. \\
Beafore procede to the acquisition, is needed to ensuring no alteration will take place, so do somethings like enable write blockers for physical storage devices, disalbe connection and sinking. particulary complex is maintain  integrity on live systems (e.g., using remote collection methods that minimize data alteration risks)


\subsection{Acquisition}
The act of performing a forensic copy (so a bit-by-bit copy) of the original data with the goal of ensure that the acquired data is a faithful replica of the source so to maintaining data integrity. \\
There 2 two main acquisition methods:
\begin{itemize}
    \item \textbf{Static:} When the system is powered down, it's se most commod method for acquiring data from hard drives and extenral memory
    \item \textbf{Live:} The system is running and it's needed to deal with volatile data like RAM, network connections, or running processes.
\end{itemize} 


\subsection{Evaluation}

