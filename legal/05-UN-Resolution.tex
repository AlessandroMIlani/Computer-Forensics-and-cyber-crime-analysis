\chapter{UN Resolution on Cybercrime}

\section{Background and Objectives of the Resolution}
The primary objective of this resolution is to initiate the drafting of a global treaty aimed at combating cybercrime through multilateral negotiations. The title, "Countering the use of information and communications technologies for criminal purposes," signifies a step towards developing an international convention on cybercrime, targeting the implementation of concrete measures against this form of crime. \bigskip

The resolution establishes a dedicated Committee responsible for drafting a comprehensive convention. This process is expected to be transparent, involving a wide range of stakeholders, such as developing countries, intergovernmental organizations, and field experts.

\section{Impact on International Cybersecurity Policies and Practices}

\subsection{UN Resolution (26 May 2021)}
This resolution could significantly impact international cybersecurity policies, promoting greater cooperation and harmonization among states. By establishing shared minimum standards through an international cybercrime treaty, it fosters a more unified regulatory framework and enhances investigative cooperation across different jurisdictions. \bigskip

The adoption of the resolution required various amendments and compromises, highlighting the importance of balancing national and supranational interests while ensuring inclusivity and transparency.

\subsection{Human Rights and Privacy Risks}
One of the central issues discussed in the UN report is the need to balance cybersecurity measures with the protection of human rights, particularly privacy and data protection. Although there is a consensus on the importance of cybersecurity, the UN GGE report warns against overreach, as unregulated measures may infringe on civil liberties.

\subsection{Emerging Threats and Supply Chain Integrity}
The report emphasizes the growing risks associated with vulnerabilities in global supply chains, especially in the ICT sector. Such vulnerabilities could lead to large-scale attacks or espionage, underlining the need for states to secure digital infrastructures and collaborate on emerging threat information.

\section{Future Legal Framework for Cybersecurity}
Cybersecurity policymakers are required to implement stringent measures to protect data during investigations. Legislative provisions should allow for periodic review and updates to cybersecurity practices, ensuring adaptability to evolving digital threats.

\section{Analysis of Key Components and Legal Implications of the
Resolution}

\subsection{Ad Hoc Committee}
The resolution establishes an Ad Hoc Committee to draft a global cybercrime convention. This committee is mandated to convene at least six times, each session lasting 10 days, beginning in January 2022. Decisions within the committee are expected to be made by consensus or, failing that, by a two-thirds majority.

\subsection{Cyber Sovereignty and Legal Boundaries}
The concept of cyber sovereignty presents a significant legal challenge. Countries such as China and Russia advocate for state control over cyberspace, potentially clashing with international norms of internet freedom and openness. The UN resolution seeks to address these tensions, but the broader debate over the degree of state control in cyberspace governance remains unresolved, necessitating a balance between sovereignty and global cooperation.

\subsection{Public-Private Cooperation and Liability}
Legal frameworks must account for the increasing role of private companies in cybersecurity. The resolution encourages collaboration between states and private entities, such as internet service providers and cybersecurity firms, to counter cyber threats. This approach raises legal questions concerning the responsibility and liability of these companies, especially when they participate in responding to or preventing cyberattacks.

\section{Jurisdictional Provisions (Article 22)}

\subsection{Territorial Jurisdiction}
State Parties must establish jurisdiction over offenses committed within their territory or on vessels or aircraft registered under their laws.

\subsection{Extended Jurisdiction}
States may also establish jurisdiction over offenses that:
\begin{itemize}
  \item Are committed against their nationals.
  \item Are committed by their nationals or stateless persons
    habitually residing within their territory.
  \item Are committed outside their territory with the intent of
    carrying out an offense within their territory, as specified in
    Article 17 of the Convention.
  \item Are committed against the State itself.
\end{itemize}

\subsection{Jurisdiction and Non-Extradition}
If the alleged offender is present within a State's territory and not extradited solely due to nationality, the State must establish jurisdiction over the offense. In cases where extradition is denied for other reasons, States may take additional measures to establish jurisdiction.

\subsection{Coordination Among States}
When a State exercising jurisdiction becomes aware of other States conducting investigations or proceedings for the same offense, authorities are encouraged to consult and coordinate their actions.

\subsection{Compatibility with International Law}
This Article affirms that the Convention does not prevent any State Party from exercising other forms of criminal jurisdiction, as permitted by its domestic law.



\section{Garlasco Case}
Digital evidence could be altered and can contain countless pieces of information. The “Garlasco” case is a clear example of this. \bigskip

Alberto Stasi was acquitted of the murder of his girlfriend, Chiara Poggi, by the Court of First Instance in December 2009, and the judgment was confirmed in the Appeal court in December 2011.

\section{Italian Case Law on Digital Forensics}
\begin{itemize}
    \item \textbf{13/08/07:} Stasi wakes up at 9, telephones Chiara Poggi, works on his thesis.
    \item \textbf{14/08/07:} Chiara Poggi died between 10:30 and 12:00.
    \item \textbf{29/08/07:} Stasi voluntarily hands over his PC to the Police.
    \item \textbf{17/12/09:} Judge Vitelli acquits Stasi of murder.
\end{itemize}
The expert report requested by the judge shows that Stasi was working on his thesis during the period when Chiara Poggi was killed.

\section{The Internet of the Human Body: Towards a Habeas Data?}
\begin{multicols}{2}
    \begin{quote}
        “If your internet thermostat's pinging servers all day, will the cops think you're a weed farm? Or just a hot yoga gym?" \\ \textit{Jonathan Zittrain}
    \end{quote}

    \begin{quote}
        “Sure, encrypt your email – while your shiny IoT toothbrush spies on you” \\ \textit{Susan Landau}
    \end{quote}
\end{multicols}

\newpage

\section{Cases}

\subsection{Connie Debate Case: Fitbit}

\begin{multicols}{2}
    \begin{quote}
        “As people continue to provide more and more personal information through technology, they have to understand we are obligated to find the best evidence, and this technology has become a part of that.”  \\
        \textit{Detective Christopher Jones - East Lampeter Township Police Department in Pennsylvania}
    \end{quote}

    \begin{quote}
        “We are entering an era of sensorveillance. People are just waking up to the fact that their smart devices are going to snitch on them and that they are going to reveal intimate details about their lives they did not intend law enforcement to have”  \\
        \textit{Andrew Ferguson, a University of the District of Columbia law professor}
    \end{quote}
\end{multicols}



\subsection{James Bate Case: Amazon Echo}

\begin{multicols}{2}
    \begin{quote}
        “The Amazon Echo device is constantly listening for the 'wake' command of 'Alexa' or 'Amazon,' and records any command, inquiry, or verbal gesture given after that point”  \\
        \textit{Search Warrant}
    \end{quote}

    \begin{quote}
        “The allegation that the Echo is possibly recording at all times without the wake word being issued is incorrect”  \\
        \textit{Answer of an Amazon Representative}
    \end{quote}
\end{multicols}

There are prloblems with the term and conditions of the Amazon Echo, as they are record data even without the wake word being issued. This is also correlated with all the problem reguardind the user of dark pattern for avoid that the use could block this kind of data collection. \\
In another case Amazon refused to give data collected by an Amazon echo to the police and use the term and conditions as exuse. \bigskip

NOTE: particularly attection for device that arrive from China (alsways check the user term and conditions).

\subsection{Ross Compton Case: Pacemaker}

\begin{multicols}{2}
    \begin{quote}
        “There is a lot of other information about things that may characterize the inside of my body that I would much prefer to keep private rather than how my heart is beating. It is just not that big of a deal” \\ 
        \textit{Judge Charles Pater}
    \end{quote}

    \begin{quote}
        “Americans shouldn't have to make a choice between health and privacy. Compelling citizens to turn over protected health data to law enforcement erodes those rights.”  \\
        \textit{Electronic Frontier Foundation Attorney Stephanie Lacambra}
    \end{quote}
\end{multicols}

Accused of frode from the incusance (hose burn), at the end of the process, the judge accept to aquire the data from the peace maker, to check the beat of the heart at the time of the incident. \bigskip

Inside our body there are a lot of data that could be used for the investigation, but the question is: how much of this data could be used for the investigation? 

\section{Facial Recognition Biometric Border}

\begin{multicols}{2}
    \begin{quote}
        “U.S. Customs and Border Protection says it will delete the live photos captured at the gate within 14 days for citizens, and that it only uses them to verify identity by comparing them with the database photos”  \\
        \textit{CBP Privacy Impact Assessment}
    \end{quote}

    \begin{quote}
        “Face Recognition has a great potential for expansion and misuse: for example, you can subject thousands of people to face recognition when they’re walking down the sidewalk without their knowledge”  \\
        \textit{Senior Policy Analyst, ACLU - Jay Stanley}
    \end{quote}
\end{multicols}

In europe there is a law that proibit the use of facial recognition in real time for the identification of the people, but in the US this is not the case. \bigskip

\section{Categories of Law Enforcement Activity}

\begin{itemize}[itemsep=0pt]
    \item Situations involving officers observing an ongoing crime (\textbf{FaceFirst})
    \item Situations involving officers investigating a past crime \textbf{(KeyCrime$^3$)}
    \item Situations involving officers predicting a future crime (\textbf{PredPol}, used in US, illegal in EU for AI act) \\ Tool for analysze crime that could happen in the future baed on the use of bigdata from the db of the law enforcement and social media data. \\ The problem for the critisim is that if you predict a crime in a certain area, the police will go in that area and this could lead to a self-fulfilling prophecy.
\end{itemize}

\section{“Police” Directive}

The Pocile Directive can be seen as a GDPR for the police. It is composed by 5 pillars:

\begin{itemize}[itemsep=0pt]
    \item Fairly, lawful, and adequate data processing during criminal investigations or to prevent a crime
    \item Clear distinction of various categories of data subjects in a criminal proceeding (investigated person, person convicted, victim of crime, third parties to the criminal offense)
    \item Prohibit measures that produce adverse legal effects for the data subject based solely on automated processing of personal data
    \item Implementation of privacy by design and by default mechanisms to ensure protection of data subject rights and minimal processing
    \item Cooperation with relevant supervisory authorities, providing all necessary information for their duties
\end{itemize}

\section{Automated Processing (Article 11 - “Police Directive”)}

Automated processing is forbidden unless:
\begin{itemize}[itemsep=0pt]
    \item \textbf{Human Intervention:} Automated processing is forbidden unless:
    \begin{enumerate}[itemsep=0pt,  label=\roman*.]
        \item There is human intervention
        \item Produce an adverse legal effect concerning the data subject
        \item Is authorized by UE or Member states
        \item Provides appropriate safeguards for the rights and freedoms of the data subject
    \end{enumerate}
    \item \textbf{Profiling:} Profiling that results in discrimination against natural persons on the basis of special categories of personal data referred to in Article 10 shall be prohibited, in accordance with Union law.
    
    \item \textbf{Access to electronic:} Legislation permitting public authorities to gain access to the contents of electronic communications on a generalised basis must be regarded as compromising the essence of the fundamental right to respect for one’s private life, as guaranteed by Article 7 of the Charter
\end{itemize}

\section{Transparency, Retention, and Enforcement}

\begin{itemize}[itemsep=0pt]
    \item \textbf{Transparency}: Tools based on big data for law enforcement purposes is checked by the law enforcement authority prior to final purchase and can be verified for its suitability, correctness and security, bearing in mind that transparency and accountability are limited by proprietary software
    \item \textbf{Retention}: While EU legal framework on data retention still lacking and we are waiting the Guidelines, safeguards are required for data retention to be lawful according to the ECJ case law are: 
    \begin{enumerate}[itemsep=0pt, label=\roman*.]
        \item serious crime;
        \item necessary and proportionate retention measure;
        \item national authorities’ access should meet certain data protection safeguards
    \end{enumerate}
    
    \item \textbf{Enforcement}: Member States shall lay down the rules on penalties applicable to infringements of the provisions adopted pursuant to this Directive and shall take all measures necessary to ensure that they are implemented. The penalties provided for shall be effective, proportionate and dissuasive
\end{itemize}

\section{Privacy vs Security}

\textit{What happens if security prevails over privacy?} - Netflix case

\section{Budapest Convention on Cybercrime - Overview}

The Budapest Convention on Cybercrime was issued by the Council of Europe on November 23, 2001.  
Italy ratified the Convention with Law n. 48 on March 18, 2008, and it was published in the Official Gazette on April 4, 2008.