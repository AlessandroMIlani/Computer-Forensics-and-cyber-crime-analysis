\chapter{Legal Introduction}

Beafore the technology was between us (classic telephon call), but with the advanced of the information technology, 
the technology is now about us (facial recognition, social media, etc). \\
Example: The IA act say taht is not possible utilize IA for real time facial recognition without an "important" reason.\\
This generation use technology also for be profiled by an alghoritm for varius scope and not only for comunicate.   \\

The "technology was between us" was simpler and the only problem was to cehck if there is a conversetion and intercept it with a good quality. 
Now, we have the problem of quantity. If we need to analyze data from milions of people we can end to vaiolate fundamental rights and make mistaks
(even with OSINT). \\ 
Surovieky theory: collective intelligence, the point is that if you have 10k persons say that the restuarant is good and only 10 that say it's a froud,
 the collective intelligence say that the majority have right. In Forensics this can not be applied because you need to be 100\% secure of what you have.
(if in a trial you have only the 1\% that a person can't be guily you need to be in favor of him) [find better term]   

In the technology in Us, and the advance of IA is important that the law split waht is uman and what is not (like be transparent when a content is AI gen and when not)

"Tesla case" whan there is an incident it need to understand in the percent of error that is from tesla and the percent from the partners. 


\section{GDPR aand alghoritm bias}

Art.22 of GDPR, say that you always need to have human in the decision process

\section{Example - Lex Machina}
Tool that analyze all the legal case from a giurisdiction (like France) and classify 
all the case in different categorys.
So if you have a case X in Paris with judge Y, you have 60\% possibility to win. If the Attorny ins Z, the probability is 80\%.

\section{Example - Compas}
alghoritm that help the jugde defice is the person can commit other crime or not 
and so decide it need to stay in jail or get a reduce in the sentence (sconto della pena)

- A False positive in digital forensics can change people's lives.



%% 1. Introduction to the Course
%% 2. Foundations of Digital Forensics
%% 3. Cybercrime Convention
%% 4. Resolution Adopted by the UN General Assembly on 26 May 2021
%% 5. Garlasco case
%% 6. IoT and Digital Forensics
%% 7. Digital Forensics and Territorial Principle
%% 8. Remote Forensics: Hacking Team Case
%% 9. Digital Forensics and Artificial Intelligence
%% 10. The New Frontier of Digital Forensics in the AI Context
%% 11. Case studies and practical applications