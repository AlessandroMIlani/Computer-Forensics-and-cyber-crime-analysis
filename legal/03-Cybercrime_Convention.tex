\chapter{Convention on Cybercrime}

\section{E-commerce on Dark Web}
The Dark Web provides a platform for buyers and sellers to engage in e-commerce transactions, often involving illicit goods and services. This anonymous marketplace operates with the same principles as traditional e-commerce, but with heightened security and privacy measures to conceal identities. \\ Vendors on the Dark Web offer a wide range of products, from drugs and weapons to stolen data and hacking services. Buyers can browse listings, read reviews, and complete purchases using cryptocurrencies, all while maintaining a high degree of anonymity.

\subsection{Silk Road}
Silk Road, often referred to as the "eBay of drugs," was an online marketplace that facilitated the sale of a wide range of illegal substances, including narcotics and controlled substances. At its peak in 2013, Silk Road had a reported annual revenue of \$89.7 million

\begin{itemize}[itemsep=0pt]
  \item \textbf{Combining Tor, PGP, and Bitcoin}: Ross Ulbricht leveraged the anonymity of Tor, the encryption of PGP, and the decentralized nature of Bitcoin to create the Silk Road marketplace.
  \item \textbf{Bitcoin-only Payments}: Silk Road required all transactions to be conducted using Bitcoin, providing an added layer of anonymity and making it harder to trace purchases.
  \item \textbf{User-friendly Interface}: Silk Road featured a well-designed interface that allowed users to easily navigate the site and leave feedback on their transactions.
  \item \textbf{Intermediary Role:} Silk Road acted as an intermediary, handling the payment processing nad logistics of shipping items purchased on the marketplace.
\end{itemize}

\subsubsection{Silk Road Investigations}

\begin{itemize}[itemsep=0pt]
  \item \textbf{Operation "Marco Polo":} Undercover agents from DEA and Secret Service involved in exorting money from Ulbricht and attempting to threaten him.
  \item \textbf{Silk Road's Scpoe:} At its peck:
    \begin{itemize}[itemsep=0pt, topsep=0pt]
      \begin{multicols}{3}
        \item 950,000 registered users
        \item 1.2 million transactions
        \item \$79 million in commissions
      \end{multicols}
    \end{itemize}
  \item \textbf{Incriminating Errors:} Ulbricht made several mistakes that led to his arrest, including using his real email address and have counterfeit documents delivered to his home.
\end{itemize}

\subsubsection{Charges Against Ross Ulbricht}
\begin{itemize}[itemsep=0pt]
  \item \textbf{SUmmary of Charges:} Ulbricht faced 7 key charges, including drug trafficking, money laundering, and computer hacking, which were consolidated into 3 main counts against him
  \item \textbf{Legal Process:} The trial lasted just 13 days and resulted in Ulbricht's conviction and life sentence, plus \$180 million in damages
\end{itemize}

The investigation was possible because involve US citizen, US platform, and US law enforcement. In another country, the same investigation would have been more difficult.

\section{History and objectives of the Convention on Cybercrime (Budapest Convention)}

The convention involve 65+ states, and its main objectives are:
\begin{itemize}[itemsep=0pt]
  \item Harmonizing national laws on cybercrime
  \item Improving investigative techniques
  \item Increasing international cooperation
\end{itemize}

\subsection{Overview of the Convention}

\subsubsection{Timeline and Ratification}

\begin{boxH}
  The Council of Europe $\neq$ Europe Union, and it's composed by state that are part of the European continent, and its conventions are open to non-European countries.
\end{boxH}

\begin{itemize}[itemsep=0pt]
  \item Full Adoption: Committee of Ministers of the Council of Europe, November 8, 2001
  \item Signature: Budapest, November 23, 2001
  \item Entry into Force: July 1, 2004
  \item Participating States (as of April 2023):
    \begin{itemize}[itemsep=0pt]
      \item 68 States have ratified
      \item 2 States signed but not ratified (Ireland, South Africa), so for them is not binding
    \end{itemize}
\end{itemize}

\subsubsection{Criticism and Opposition}
\begin{itemize}[itemsep=0pt]
  \item \textbf{India:} Initially refused to adopt due to non-participation in drafting
  \item \textbf{Reconsideration} (since 2018): Surge in cybercrime, but concerns about data sharing with foreign agencies remain
  \item \textbf{Russia}: Rejected due to concerns about sovereignty, limited cooperation in international investigations, even after some articles were revised to address these concerns
\end{itemize}

\subsubsection{New Global Cybercrime Treaty (UN, August 8, 2024)}

\textbf{Content}:
\begin{itemize}[itemsep=0pt]
  \item Criminalization of unauthorized access to information systems
  \item Crimes related to online child exploitation and non- consensual explicit content distribution
\end{itemize}

\textbf{Criticism}
\begin{itemize}[itemsep=0pt]
  \item Concerns over human rights and press freedom
  \item Issues with data privacy and overly broad definitions of cybercrime
\end{itemize}

\subsection{Aim of the Convention}
The convention aims to assist in combating crimes that are inherently linked to the use of technology, where devices serve as both the tool for committing the crime and the target of the crime
\begin{boxH}
  \textbf{Note:} there are differences when a crime is commited through a technology tool, and when the technology is the target of the crime.
\end{boxH}
as well as crimes where technology is used to enhance other criminal activities, such as fraud. It provides guidelines for countries to develop domestic cybercrime laws and acts as a foundation for international cooperation between parties.\\The \textbf{first additional protocol} focuses on criminalizing the dissemination of racist and xenophobic material through computer systems, along with threats and insults motivated by racism and xenophobia.\\ The \textbf{second additional protocol} establishes common international rules to enhance cooperation on cybercrime, particularly in the collection of electronic evidence for criminal investigations and proceedings.

\subsection{Key points}

\subsubsection{Original Convention (1 July 2004)}
\begin{itemize}[itemsep=0pt]
  \item The convention covers:
    \begin{itemize}[itemsep=0pt]
      \item The criminalisation of conduct: ranging from illegal access, data and systems interference to computer-related fraud and dissemination of child abuse material;
      \item Procedural powers to investigate cybercrime and secure electronic evidence in relation to any crime;
      \item Efficient international cooperation between parties.
    \end{itemize}
  \item Parties are members of the Cybercrime Convention Committee and share information and experience, assess implementation of the convention or interpret the convention through guidance notes.
  \item Of the 27 Member States, 26 have ratified the convention - Ireland has signed but not yet ratified it.
\end{itemize}

\subsubsection{Additional Protocol 1 (1 March 2006)}

This protocol extends the scope of the convention to cover xenophobic and racist propaganda disseminated through computer systems, providing more protection for victims. It furthermore:

\begin{itemize}[itemsep=0pt]
  \item  Reinforces the legal framework through a set of guidelines for criminalising xenophobia and racist propaganda in cyberspace;
  \item Enhances the ways and means for international cooperation in the investigation and prosecution of racist and xenophobic crimes online.
\end{itemize}

\subsubsection{Additional Protocol 2 (8 August 2024)}
This protocol aims to further enhance international cooperation.\\
It addresses the particular challenge of electronic evidence relating to
cybercrime and other offences being held by service providers in
foreign jurisdictions, but with law enforcement powers limited to
national boundaries.\\
Its main features are:
\begin{itemize}[itemsep=0pt]
  \item A new legal basis permitting a direct request to registrars in other jurisdictions to obtain domain name registration information
  \item A new legal base permitting direct orders to service providers in other jurisdictions to obtain subscriber information
  \item Enhanced means for obtaining subscriber information and traffic data through "government to government" cooperation
  \item Expedited cooperation in emergency situations including the use of joint investigation teams and joint investigations. Key points of the Additional Protocol 2 (8 August 2024)
\end{itemize}

\section{Harmonization of national laws and international cooperation}

\subsection{Internetional Cooperation Provisions}
\begin{itemize}[itemsep=0pt]
  \item \textbf{Coooperation Principle:} Parties are to cooperate "to the widest extent possible" in investigating electronic evidence.
  \item \textbf{Expedited Mutual Assistance:} Issue with Current Mechanisms: Mutual assistance requests are often slow and take months.
  \item \textbf{Convention solution:} Allows for expedited requests using "expedited means of communication" and Expedited means must provide adequate levels of security and authentication. (So can be used encrypted physical device for data transfer)
  \item \textbf{Voluntary Information Sharing:} Parties may share information without a formal request if it would assist in investigations or help the receiving party with any related offences.
\end{itemize}

\subsection{Mutual Assistance Provisions}
\begin{itemize}[itemsep=0pt]
  \item  \textbf{Procedural Powers for Assistance:}
    \begin{itemize}[itemsep=0pt]
      \item Expedited Preservation of stored computer data.
      \item Expedited Disclosure of traffic data.
      \item Real-time Collection of traffic data and interception of content data: parties provide assistance according to domestic laws and applicable treaties, subject to any reservations.
    \end{itemize}
  \item \textbf{Art 23 - General Cooperation Principle:}
    \begin{itemize}[itemsep=0pt]
      \item Mutual assistance "to the widest extent possible" for:
      \item Cyber-related offences.
      \item Collection of electronic evidence for any criminal offence.
    \end{itemize}
  \item \textbf{Restrictions:}
    \begin{itemize}[itemsep=0pt]
      \item Cooperation may be restricted in cases of:
      \item Extradition.
      \item Mutual assistance regarding real-time collection of traffic data.
      \item Interception of content data
    \end{itemize}
\end{itemize}

\subsection{24/7 Network for Immediate Assistance}
\textbf{Provision for Constant Availability:}
\begin{itemize}[itemsep=0pt]
  \item Each party must designate a contact point available 24/7
  \item Purpose: Provide immediate assistance for cybercrime investigations, proceedings, or the collection of electronic evidence
  \item Based on the G8 network of contact points model
\end{itemize}
\textbf{Significance:} aims to expedite the processing of urgent
mutual assistance requests, overcoming current delays in
traditional bureaucratic channels

\section{Legal measures against computer-related fraud and forgery}

\subsection{Criminalization of Fraud and Computer-related Forgery}

The Convention requires State Parties to criminalize specific conducts, including fraud and forgery carried out through computer systems. This includes, for instance, digital document forgery and fraud involving the use of electronic data to deceive or gain financial benefits. \\\textbf{Computer-related fraud involves using computers to gain economic benefits} through deceit, while \textbf{forgery includes altering or creating digital documents} with the intent to mislead. 


\subsubsection{Computer-Related Forgery}
Each Party shall adopt such legislative and other measures as may be necessary to establish as criminal offences under its domestic law, when committed intentionally and without right, the input, \textbf{alteration}, \textbf{deletion}, or s\textbf{uppression of computer data}, resulting in inauthentic data with the intent that it be considered or acted upon for legal purposes as if it were authentic, regardless whether or not the data is directly readable and intelligible. A Party may require an intent to defraud, or similar dishonest intent, before criminal liability attaches. \\
This kind of alteration it's often seen as less grave compared to the forgery fo a physical document, becasue at a cercain level is easier to do (like copy the image of a firm on a document).

\subsubsection{Computer-Related Fraud}

Each Party shall adopt such legislative and other measures as may be necessary to establish as criminal offences under its domestic law, when committed intentionally and without right, the causing of a loss of property to another person by:

\begin{itemize}
\item any input, alteration, deletion or suppression of computer data,
\item any interference with the functioning of a computer system
\end{itemize}

With fraudulent or dishonest intent of procuring, without right, an economic benefit for oneself or for another person.

\subsection{Procedural Law Tools}

To effectively address these crimes, the Convention introduces procedural law tools that allow for quicker and more effective investigations. For example, \textbf{expedited preservation of volatile data} and \textbf{seizure of information} are crucial tools for gathering evidence in investigations against digital fraud and forgery. The Convention mandates that criminal justice authorities must be able to use effective means, such as:

\begin{itemize}[itemsep=0pt]
    \item \textit{Search and seizure}
    \item \textit{Access to stored data in computer systems}
\end{itemize}

These tools must be applicable regardless of the type of crime involved.

\section{Procedural powers for law enforcement}

\subsection{Synopticon and Omnipticon}

A \textbf{synoption} world, the many watch the few, so there is possibility to control the information, but in the digital world, in a \textbf{omnipticon} world, the many watch the many, so tit's extremly complex try to control the information. \\
The tv, as a synoption point of view have also keep the role of an "official information soruce", but in the last years, with the diffusion in the use of youtube or similar platform, also this role is starting to crumble. \\
In this new scenario, we need to consider how is a mess, from a legal stand point manage all the video.

\begin{boxH}
  The \textbf{Main Concert} for private citizens and public administration using cloud techmologies is not so much the possibile increase in "cyber" fraud or crime, than the loss of control over one's data, for privacy reason and \textbf{digital investigaion purposes}
\end{boxH}

\subsection{Some articles}

\begin{boxH}
  Articles from 14 to 20 are not very important for the Exam
\end{boxH}

\subsubsection{Scope of Procedural Provisions (Article 14)}

Each Party must adopt \textbf{legislative measures} to define the \textit{powers and procedures} for specific criminal investigations or proceedings.\\ The provisions apply to:

\begin{itemize}[itemsep=0pt]
    \item Offenses covered by the Convention,
    \item All other offenses committed through \textbf{computer systems},
    \item All \textbf{electronic evidence} related to any crime.
\end{itemize}

\subsubsection{Conditions and Safeguards (Article 15)}

The application of \textbf{powers and procedures} must ensure \textit{adequate protection of human rights}, following national law and international conventions (e.g., the \textbf{European Convention on Human Rights}). Conditions and safeguards include:

\begin{itemize}[itemsep=0pt]
    \item \textbf{Judicial or independent supervision},
    \item Consideration of \textit{proportionality},
    \item Protection of the \textit{rights of third parties}.
\end{itemize}

\subsection{Expedited Preservation of Stored Data (Article 16)}

Authorities must be able to \textbf{order or obtain the rapid preservation} of specific computer data, particularly if there is reason to believe that the data is vulnerable to \textit{deletion or modification}. This order may require the data's custodian to preserve the data for up to \textbf{90 days}, which is extendable as needed. \\ In Italy the authorities can access the list of the web access of the previus 5 years of a person if it's under investigation.

\subsubsection{Expedited Disclosure of Traffic Data (Article 17)}

To ensure data preservation, authorities can demand rapid disclosure of traffic data to identify service providers and communication pathways, even if multiple providers are involved in the transmission.

\subsubsection{Production Order (Article 18)}

Each Party shall adopt such legislative and other measures as may be necessary to empower its competent authorities to order:

\begin{enumerate}[itemsep=0pt]
    \item A person in its territory to submit specified computer data in that person’s possession or control, which is stored in a computer system or a computer-data storage medium;
    \item A service provider offering its services in the territory of the Party to submit subscriber information relating to such services in that service provider’s possession or control.
\end{enumerate}

\subsubsection{Subscriber Information (Article 18)}

For the purpose of this article, the term \textit{subscriber information} means any information contained in the form of computer data or any other form that is held by a service provider, relating to subscribers of its services other than traffic or content data and by which can be established:

\begin{enumerate}[itemsep=0pt]
    \item The type of communication service used, the technical provisions taken there to, and the period of service;
    \item The subscriber’s identity, postal or geographic address, telephone and other access numbers, billing and payment information, available on the basis of the service agreement or arrangement;
    \item Any other information on the site of the installation of communication equipment, available on the basis of the service agreement or arrangement.
\end{enumerate}



\subsubsection{Search and Seizure of Stored Computer Data (Article 19)}

\begin{boxH}
  One of the least applicated article.
\end{boxH}

Each Party shall adopt such legislative and other measures as may be necessary to empower its competent authorities to search or similarly access:

\begin{enumerate}[itemsep=0pt]
    \item A computer system or part of it and computer data stored therein; and
    \item A computer-data storage medium in which computer data may be stored in its territory.
\end{enumerate}

\subsubsection{Real-time Collection of Traffic Data (Article 20)}

Authorities can \textbf{collect} or record \textbf{traffic data in real time}, either directly or by requiring service providers to assist in the collection. (convention is usually limited to telco and ISP of the country)

\subsubsection{Interception of Content Data (Article 21)}

For serious offenses, \textbf{authorities may intercept or record the content of specific communications in real time}, either directly or through the cooperation of service providers. 

\subsubsection{Mutual Assistance (Article 25)}

The Parties shall afford one another \textbf{mutual assistance to the widest extent possible} \textbf{for the purpose of investigations or proceedings} concerning criminal offences related to computer systems and data, or for the collection of evidence in electronic form of a criminal offence.

\subsubsection{Expedited Preservation of Stored Computer Data (Article 29)}

A Party may request another Party to order or otherwise obtain the expeditious preservation of data stored by means of a computer system, located within the territory of that other Party and in respect of which the requesting Party intends to submit a request for mutual assistance for the search or similar access, seizure or similar securing, or disclosure of the data.

\subsubsection{Expedited Disclosure of Preserved Traffic Data (Article 30)}

Where, in the course of the execution of a request made pursuant to Article 29 to preserve traffic data concerning a specific communication, the requested Party discovers that a service provider in another State was involved in the transmission of the communication, the requested Party shall expeditiously disclose to the requesting Party a sufficient amount of traffic data to identify that service provider and the path through which the communication was transmitted.


\subsection{Article 32. Solution to Russia Concerns}
A party may, without the authorisatoin of another Party:
\begin{itemize}
  \item Acces publicy available stored computer data, regardless of where the data is located geographically
  \item Access or receive, through a computer system in its territory, stored computer data located in another Party, if the Party obtains the lawful and voluntary consent of the person who has the lawful authority to disclose the data to the Party through that computer system
\end{itemize}

\subsubsection{Conseguences of Article 32b}
\begin{boxH}
LEAs routinely requestand are provided with data from from foreign service providers, withoutformal inter-State process such as mutual legal assistance (MLA). Ebay and Facebook hace dedicated portals for facilitating such Exhcaneges.
\end{boxH}

There are 5 proposed implementation from the Council of Europe for the article 32b:
\begin{enumerate}[itemsep=0pt]
  \item "Transborder access with consent without the limitation to data stored 'in another Party'"
  \item "Transborder access without consent but with lawfully obtained credentials"
  \item "Transborder access without consent in good faith or in exigent or other circumstances"
  \item "When the data is lawfully accessible or available from the initial system™
  \item If the location of the data is not known, but the person having the power of disposal of the data is physically on the territory of, or a national of the searching Party, the LEA of this Party may be able [to] search or otherwise access the data
\end{enumerate}


\section{Some additional Legal framework}

\begin{boxH}
  An \textbf{EU regulation} is someting that is directly applicable in all the EU member states, so it's not necessary to be trasposed in the national law of the member states.
\end{boxH}

\begin{itemize}
  \item \textbf{Regulation (EU) 2023/1543} of the European Parliament and of the Council of 12 July 2023 on European Production Orders and European Preservation Orders for electronic evidence in criminal proceedings and for the execution of custodial sentences following criminal \\ \href{https://eur-lex.europa.eu/legal-content/EN/TXT/?uri=CELEX%3A32023R1543}{Text of the Regulation}
  \item \textbf{Directive (EU) 2023/1544} of the European Parliament and of the Council of 12 July 2023 laying down harmonised rules on the designation of designated establishments and the appointment of legal representatives for the purpose of gathering electronic evidence in criminal proceedings \\ \href{https://eur-lex.europa.eu/legal-content/EN/TXT/?uri=CELEX%3A32023L1544}{Text of the Directive}
\end{itemize}

\begin{boxH}
  An \textbf{EU directive} is a legal act that sets out a goal that all EU countries must achieve. However, it is up to the individual countries to devise their own laws on how to reach these goals. (the directive 2023/1544 must be trasposed in the national law by 18 February 2026)
\end{boxH}                                                                                                                                                                                                                                                                                                                                                                                                                                                                                                                                                                                                                                                                                                                                                                                                                                                                                                                                                                                                                                                                            